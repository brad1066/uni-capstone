\documentclass[12pt, a4paper,twoside]{report}

%% Every LaTeX document begins with a preamble, which loads packages and
%% defines various settings to make the document look right. Mostly,
%% you can ignore everything in this template before \begin{document} on
%% line 74

\usepackage{mathtools,amsthm} % Enable useful mathematical symbols/environments
\usepackage{graphicx} % Enable graphics
\usepackage{fancyhdr,titlesec,microtype} % enable various formatting commands
\usepackage[margin=2.5cm]{geometry} % Set margin size
\usepackage{palatino} % Set the font
\usepackage[latin1]{inputenc} % Allow you to input accents, umlauts and other characters
\usepackage[T1]{fontenc} % Lets LaTeX print a wider array of characters

\usepackage{xcolor} % Enable coloured elements
\definecolor{mypurple}{HTML}{622567} %%% Purple
\definecolor{myred}{HTML}{D55C19} %%%EssexOrange
\definecolor{myblue}{HTML}{007A87} %%%Seagrass

% For technical reasons, hyperref should be loaded after all other packages
\usepackage[colorlinks,linkcolor=myblue,citecolor=mypurple]{hyperref}

\renewcommand{\baselinestretch}{1.5} % 1.5 line spacing

% Define \begin{theorem}, \end{theorem}, etc.
\theoremstyle{plain} % The following will be italicised
\newtheorem{theorem}{Theorem}[chapter]

\theoremstyle{definition} % The following environments will not use italics
\newtheorem{definition}[theorem]{Definition}
\newtheorem{example}[theorem]{Example}

\numberwithin{equation}{chapter}

% Fancy headings
\pagestyle{fancy}
\setlength{\headheight}{15pt}
\fancyheadoffset[LE,RO]{0pt}
\renewcommand{\chaptermark}[1]{\markboth{#1}{}}
\renewcommand{\sectionmark}[1]{\markright{\thesection\ #1}}
\fancyhf{}
\fancyhead[LE]{\makebox[0pt][l]{\thepage}\hfill\leftmark}
\fancyhead[RO]{\rightmark\hfill\makebox[0pt][r]{\thepage}}
\fancypagestyle{plain}{%
    \fancyhead{} % get rid of headers
    \renewcommand{\headrulewidth}{0pt} % and the line
}

% Fancy chapter numbers
\titleformat{\chapter}[display]
    {\normalfont\bfseries\color{myred}}
    {\filleft\hspace*{-60pt}%
        \rotatebox[origin=c]{90}{%
            \normalfont\color{black}\Large%
            \textls[180]{\textsc{\chaptertitlename}}%
        }
        \hspace{10pt}%
        {\setlength\fboxsep{0pt}%
            \colorbox{myred}{\parbox[c][3cm][c]{2.5cm}{%
                \centering\color{white}\fontsize{80}{90}\selectfont\thechapter}%
            }
        }
    }
    {10pt}
    {\titlerule[2.5pt]\vskip3pt\titlerule\vskip4pt\LARGE\sffamily}

\begin{document} % Start your document

%%%%%%%%%%%% BEGIN TITLE PAGE %%%%%%%%%%%%

\thispagestyle{empty} % For the title page, no header / footer

\noindent
    \begin{minipage}{0.1\textwidth}
    \includegraphics[height=4.5em]{essex.png}
    \end{minipage}
    \hfill
    \begin{minipage}{0.5\textwidth}
    \begin{center}
        \renewcommand\familydefault{\sfdefault}
        \fontfamily{phv}\selectfont
        {\large School of Computer Science and Electrical Engineering}
    \end{center}
    \end{minipage}

\begin{center}
    \noindent\textcolor{myred}{\rule{\linewidth}{4.8pt}}
    
    \vspace{2em}
    \noindent {\LARGE \sc Capstone Project Dissertation}
    
    \vspace{3em}
    \noindent {\Huge{\color{myblue} GoLearn - Learning Resource Management System}}
    
    \vspace{3em}
    \noindent {\Large \bf Bradley Beasley}
    \vfill
    \noindent {\Large {Supervisor:} {\color{mypurple} \bf Dr Alexandoros Voudoris}}
    
    \vspace{0.5em}
    \noindent\textcolor{myred}{\rule{\linewidth}{4.8pt}}
    
    \vspace{2em}
    {\Large \today }
    
    {\Large Colchester}
\end{center}

\clearpage

%%%%%%%%%%%% END TITLE PAGE %%%%%%%%%%%%

\tableofcontents

%%%%%%%%%%%% Introduction %%%%%%%%%%%%

\chapter{Introduction}\label{ch:introduction}

%%%%%%%%%%%% Background %%%%%%%%%%%%

\chapter{Background}\label{ch:background}

\section{Similar Services}\label{sec:similarservices}

There are various LRMS's available in the market, with some of the more popular
one being:

\begin{description}
    \item[Skillshare] {}
    \item[Udemy] {}
    \item[LearnDash] {}
    \item[Moodle] {}
\end{description}

\section{Tech Stacks}\label{sec:techstacks}

A `Tech Stack' is the term used to describe the combination of programming
languages, frameworks, libraries, and tools that are used to build a software
application. The choice of a tech stack is crucial as it can affect the
performance, scalability, and maintainability of the application. The following
are some of the popular tech stacks used in web development:

\begin{description}
    \item[LAMP/LEMP Stacks] {}
    \item[MEAN/MERN/MEVN Stacks] {}
\end{description}

\section{Databases and ORMs}\label{sec:databasesorms}

what is a database.
what is an ORM
popular databases and ORMs

%%%%%%%%%%%% Technical Specification %%%%%%%%%%%%

\chapter{Technical Specification}\label{ch:techspec}

\section{Framework}\label{sec:framework}

Which stack was chosen and why
which framework was chosen within that stack and why

\section{Database and ORM}\label{sec:databaseorm}

Which database was chosen and why
which ORM was chosen and why

%%%%%%%%%%%% Implementation %%%%%%%%%%%%

\chapter{Implementation}\label{ch:implementation}

\section{Authentication}\label{sec:authentication}

\section{User Management}\label{sec:usermanagement}

\section{CRUD Operations}\label{sec:crudoperations}

\section{Data Loading}\label{sec:dataloading}

%%%%%%%%%%%% Testing %%%%%%%%%%%%

\chapter{Testing}\label{ch:testing}

\section{Unit Testing}\label{sec:unittesting}

\section{Acceptance Testing}\label{sec:acceptancetesting}

%%%%%%%%%%%% Conclusion %%%%%%%%%%%%

\chapter{Conclusions}\label{ch:conclusion}

\bibliographystyle{plain}
\bibliography{report}
\footnotetext{
    \textbf{Information}
    \emph{This document was written in LaTex and compiled with PDFTex and
        BibTex}
}

\end{document}

