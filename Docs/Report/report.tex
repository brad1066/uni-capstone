\documentclass[12pt, a4paper,twoside]{report}

%% Every LaTeX document begins with a preamble, which loads packages and
%% defines various settings to make the document look right. Mostly,
%% you can ignore everything in this template before \begin{document} on
%% line 74

\usepackage{mathtools,amsthm} % Enable useful mathematical symbols/environments
\usepackage{graphicx} % Enable graphics
\usepackage{fancyhdr,titlesec,microtype} % enable various formatting commands
\usepackage[margin=2cm]{geometry} % Set margin size
\usepackage{palatino} % Set the font
\usepackage[latin1]{inputenc} % Allow you to input accents, umlauts and other characters
\usepackage[T1]{fontenc} % Lets LaTeX print a wider array of characters

\usepackage{xcolor} % Enable coloured elements
\definecolor{mypurple}{HTML}{622567} %%% Purple
\definecolor{myred}{HTML}{D55C19} %%%EssexOrange
\definecolor{myblue}{HTML}{007A87} %%%Seagrass

% For technical reasons, hyperref should be loaded after all other packages
\usepackage[colorlinks,linkcolor=myblue,citecolor=mypurple]{hyperref}

\renewcommand{\baselinestretch}{1.25} % 1.5 line spacing

% Define \begin{theorem}, \end{theorem}, etc.
\theoremstyle{plain} % The following will be italicised
\newtheorem{theorem}{Theorem}[chapter]

\theoremstyle{definition} % The following environments will not use italics
\newtheorem{definition}[theorem]{Definition}
\newtheorem{example}[theorem]{Example}

\numberwithin{equation}{chapter}

% Fancy headings
\pagestyle{fancy}
\setlength{\headheight}{15pt}
\fancyheadoffset[LE,RO]{0pt}
\renewcommand{\chaptermark}[1]{\markboth{#1}{}}
\renewcommand{\sectionmark}[1]{\markright{\thesection\ #1}}
\fancyhf{}
\fancyhead[LE]{\makebox[0pt][l]{\thepage}\hfill\leftmark}
\fancyhead[RO]{\rightmark\hfill\makebox[0pt][r]{\thepage}}
\fancypagestyle{plain}{%
    \fancyhead{} % get rid of headers
    \renewcommand{\headrulewidth}{0pt} % and the line
}

% Fancy chapter numbers
\titleformat{\chapter}[display]
    {\normalfont\bfseries\color{myred}}
    {\filleft\hspace*{-80pt}
        \rotatebox[origin=c]{90}{
            \normalfont\color{black}\large
            \textls[180]{\textsc{\chaptertitlename}}
        }
        \hspace{8pt}
        {\setlength\fboxsep{0pt}
            \colorbox{myred}{\parbox[c][2cm][c]{1.5cm}{
                    \centering\color{white}\fontsize{40}{45}\selectfont\thechapter}
            }
        }
    }
    {10pt}
    {\titlerule[2.5pt]\vskip3pt\titlerule\vskip4pt\LARGE\sffamily}

\begin{document} % Start your document

%%%%%%%%%%%% BEGIN TITLE PAGE %%%%%%%%%%%%

\thispagestyle{empty} % For the title page, no header / footer

\noindent
\begin{minipage}{0.1\textwidth}
    \includegraphics[height=4.5em]{essex.png}
\end{minipage}

\hfill
\begin{minipage}{0.5\textwidth}
    \begin{center}
        \renewcommand\familydefault{\sfdefault}
        \fontfamily{phv}\selectfont
        {\large School of Computer Science and Electrical Engineering}
    \end{center}
\end{minipage}

\begin{center}
    \noindent\textcolor{myred}{\rule{\linewidth}{4.8pt}}

    \vspace{2em}
    \noindent {\LARGE \sc Capstone Project Dissertation}

    \vspace{3em}
    \noindent {\Huge{\color{myblue} GoLearn --- Learning Resource Management
            System}}

    \vspace{3em}
    \noindent {\Large \bf Bradley Beasley}

    \vfill
    \noindent {\Large {Supervisor:} {\color{mypurple} \bf Dr Alexandoros
            Voudoris}}

    \vspace{0.5em}
    \noindent\textcolor{myred}{\rule{\linewidth}{4.8pt}}

    \vspace{2em}
    {\Large \today }

    {\Large Colchester}
\end{center}

\clearpage

\abstract{
    There are plenty of services and programs out there for allowing Teachers
    to upload resources and for their learners to access them. Most of these
    are not purpose designed for places of learning, such as school, colleges
    and universities. Those that exist for this specific use-case tend to be
    over-engineered and can be overly complicated (Moodle is a good example
    of this). The purpose of GoLearn is to provide a simple LRM experience
    for teachers and students to interact with the same resources without a
    large amount of overhead in setting up the system: it should work out of
    the box.\\This report contains a look into methods of implementation for
    web applications in general, plus the choices in technology and design
    that were made for GoLearn.
}

%%%%%%%%%%%% END TITLE PAGE %%%%%%%%%%%%

\tableofcontents

%%%%%%%%%%%% Introduction %%%%%%%%%%%%

\chapter{Introduction}\label{ch:introduction}

%%%%%%%%%%%% Background %%%%%%%%%%%%

\chapter{Background}\label{ch:background}

\section{Similar Services}\label{sec:similarservices}

There are various LRMS's available in the market, with some of the more popular
one being:

\begin{description}
    \item[Udemy] {
        is a free to access learning site that puts a paywall over all of
        the courses. Creators can sign up to the site and create a course very
        quickly. Learners can sign up to courses (paying whatever charge is
        applied) and have unlimited access to them. They can access these from
        their own `Courses' dashboard.
        }
    \item[Skillshare] {
        is a similar service to Udemy, but is subscription based rather than
        pay per course. This means that learners can access all of the courses
        on the site for a monthly fee. Creators can create courses and upload
        them to the site, with the same access restrictions as Udemy.
        }
    \item[LearnDash\cite{learndash}] {
        is a WordPress plugin that allows you to create and sell
        courses on your WordPress site. It is a very popular plugin, with
        over 50,000 active installs and a 4.5-star rating on the WordPress
        plugin repository. It is a very powerful plugin, with a lot of
        features, but it is also very complex and can be difficult to use.\\
        Being a plugin for WP, it is very easy to customise how the plugin
        looks by modifying the theme on the site to best suit the business /
        individual's needs. The plugin allows teachers to create courses and
        students to access them, with a lot of customisation options for the
        courses.
        }
    \item[Moodle\cite{moodle}] {
        is the closest comparison that is currently available to GoLearn. It
        is a PHP based LMS that is very powerful and feature-rich. It allows
        learning institutions to design their own courses/modules and add
        specific `student' users to them. This is a very similar model to
        what GoLearn is aiming to achieve. The main difference is that Moodle
        is very complex and can be difficult to use. It is also very
        resource-intensive, requiring a lot of server resources to run
        effectively.
        }
\end{description}

\section{Tech Stacks}\label{sec:techstacks}

A `Tech Stack' is the term used to describe the combination of programming
languages, frameworks, libraries, and tools that are used to build a software
application. The choice of a tech stack is crucial as it can affect the
performance, scalability, and maintainability of the application. The following
are some of the popular tech stacks used in web development:

\begin{description}
    \item[LAMP/LEMP Stacks] {
        This stack consists of 4 major components: Linux, Apache/Nginx, MySQL
        and PHP\@. Linux is the OS that is used in the stack, with Apache or
        (E)\@nginx as the web server, MySQL as the database and PHP as the
        server-side language. This stack is very popular and is used by many
        web developers. It is used by various Frameworks (such as Laravel), as
        well as in vanilla PHP applications (no framework used).

        PHP is the programming language of choice for this kind of stack for
        websites and applications that need to render the pages dynamically,
        but don't need a lot of reactivity to user input on the page. For
        for example, a blog or a news site would be a good use case for this
        stack, as they largely need to retrieve data from a db when a page is
        loaded, and can use caching strategies easily to only make a DB request
        for cached pages after a certain period after the cached content was stored.

        On the other hand, applications that have a lot of user interaction to perform
        CRUD operations would be harder to write within this stack, as JS would have
        to be loaded and ran to submit the changed data. Other stacks (as mentioned below)
        instead rely less on page loads to perform actions, and instead are designed
        with responsive and reactive implementations in mind.
        }
    \item[MEAN/MERN/MEVN Stacks] {
        The `M' refers to the database implementation used (such as mySql, MongoDB)\ldots,
        the `E' refers to ExpressJS, which is a NodeJS framework used for all forms of
        JS applications, and is the layer between the templating framework and NodeJS to
        serve an application. The `V', `A' and `R' refer to the templating framework used
        to create the layouts used, and reference `VueJS', `AngularJS' and `ReactJS'
        respectively. The `N' refers to `NodeJS', which is a server-side JS runtime
        environment that is used to run JS code on the server.

        This stack is very popular for creating SPAs (Single Page Applications) and
        PWAs (Progressive Web Applications). These are applications that are designed
        to be very reactive to user input, and can update the page without needing to
        reload the page. This is done by using JS to update the page, rather than
        relying on the server to render the page and send it to the client. This is
        achieved by requesting information from the server about what to load and how
        to load it, and then using JS to render the page based on the information
        received. This is a very powerful stack, but can be difficult to learn and
        use effectively.
        }
    \item[Full Stack Frameworks] {
        A full stack framework is a framework that is designed to be used for both
        the front-end and back-end of an application. This means that the framework
        can be used to create the server-side code, as well as the client-side code.
        This is very useful as it means that development teams can rely on a single
        codebase for the entire application, rather than having to use multiple
        codebases. This can make development faster and easier. As well as this, it
        allows type safety and code completion to be used across the entire application
        easier than in separate front- and back-end applications.

        Some examples of full stack frameworks are NextJS and NuxtJS\@. These
        frameworks are designed to be used for both the front-end and back-end of
        an application. They are very powerful and can be used to create very complex
        applications. They each use a different templating framework (NextJS uses
        ReactJS, and NuxtJS uses VueJS). The benefit of these frameworks is that they
        allow for server-side rendering as well as client-side rendering, which can
        be very useful for SEO and performance reasons.

        One drawback of these frameworks is that there can be a steep learning curve
        to get started with them. As well as this, the codebase can get very complex
        and the boundaries between client and server code can get blurred, which can
        make it difficult to debug and maintain the codebase if it is not well organised.
        }
\end{description}

\section{Databases and ORMs}\label{sec:databasesorms}

A database is a collection of data that is stored in some structured
format. Databases are used to store data that can be accessed and
manipulated by software applications. There are many different types of
databases, each with its own strengths and weaknesses, but all fall into
one of two categories: relational or non-relational.

\begin{description}

    \item[Relational Databases] {
        store data in tables, with each table containing
        rows and columns. The columns represent the attributes of the data,
        while the rows represent individual records. Relational databases use
        SQL (Structured Query Language) to query and manipulate the data. Some
        popular relational databases include MySQL, PostgreSQL, and SQLite.

        This category of database is called relational because it allows for
        relationships to be defined between the columns in different tables.
        For example, a database for a school might have a `students' table and
        a `courses' table. The `students' table might have a column that
        references the `courses' table, which would allow for a relationship
        to be defined between the two tables. This is called a foreign key, and
        can be used to enforce that a student can only be enrolled in a course
        that exists in the `courses' table.
        }

    \item[Non-Relational Databases] {
        store data in a more flexible format, such as
        key-value pairs, documents, or graphs. Non-relational databases are
        often used when the data is unstructured or when the data model is
        likely to change frequently. Some popular non-relational databases
        include MongoDB, Cassandra, and Redis.

        This category of database is called non-relational because it does not
        enforce relationships between tables. This can make it easier to work
        with the data, as the data can be stored in a more flexible format.

        % TODO: Add more information about non-relational databases
        }

\end{description}

An ORM is a layer between the application and the database that allows the
application to interact with the database using an object-oriented interface.
These are used to simplify the process of interacting with the database, and
to make it easier to work with the data in the database. They supply an API for
developers to be able to perform CRUD operations on the database without having
to construct many (if any) queries or statements to do so.

Some examples of ORMs include Prisma, TypeORM, and Sequelize. These ORMs are
designed to be used with different databases, and have different features and
capabilities. For example, Prisma is designed to be used with PostgreSQL, MySQL,
and SQLite, and has a strong focus on type safety and code generation. TypeORM
is designed to be used with TypeScript, and has a strong focus on type safety
and code generation. Sequelize is designed to be used with MySQL, PostgreSQL,
SQLite, and MSSQL, and has a strong focus on performance and scalability.\cite{orms}

%%%%%%%%%%%% Technical Specification %%%%%%%%%%%%

\chapter{Technical Specification}\label{ch:techspec}

\section{Framework}\label{sec:framework}

Of the stacks mentioned in Section~\ref{sec:techstacks}, the NextJS framework
was chosen for the GoLearn project. This was chosen for the following reasons:

\begin{description}
    \item[ReactJS\cite{what-is-react}] {
        ReactJS is a very powerful and popular front-end framework that is
        used to create SPAs and PWAs. It is very flexible and can be used to
        create very complex applications. It is also very easy to learn and
        use, which makes it a good choice for the GoLearn project.

        As well as this, it was a framework that the author had used before,
        meaning that he could get started with the project quickly and easily
        and wouldn't have to spend a lot of time learning to use a new framework
        or architect his own solution from scratch. This proved to save a lot
        of time in the development process, as the author was able to get started
        on the project and make progress quickly.
        }
    \item[Server Actions\cite{next-server-action}] {
        As of NextJS 14, the framework included a feature called `Server Actions'.
        This allowed functions to be shared between server- and client-side code.
        Whenever a server action was called, a request is sent to the server as a
        Fetch/XHR request from the client, the server would handle the request and
        return the result to the client. This would prove to be a very useful feature,
        as it allows for CRUD operations to be ran
        }
\end{description}

\section{Database and ORM}\label{sec:databaseorm}

The ORM chosen for the GoLearn project was Prisma. This was chosen for the
following reasons:

\begin{description}
    \item[Type Safety\cite{prisma-type-safety}] {
        Prisma is designed to be used with TypeScript, and has a strong focus
        on type safety and code generation. This means that the author could
        be confident that the code he was writing was correct and that the
        database schema was being enforced correctly. This was very important
        for the GoLearn project, as it was a project that was being developed
        by a single developer, and there was no QA team to review the code.
        }
    \item[Code Generation\cite{prisma-code-generation}] {
        Prisma uses code generation to generate the database schema and the
        client-side code that is used to interact with the database. This
        means that the author could be confident that the code he was writing
        was correct and that the database schema was being enforced correctly.
        This was very important for the GoLearn project, as it was a project
        that was being developed by a single developer, and there was no QA
        team to review the code.
        }
\end{description}

Prisma can be used with many different databases, including PostgreSQL, MySQL,
and SQLite. One of the factors that was considered was the ease of creating and
maintaining the database. Originally the author had considered using a service
called `Supabase', which is a service that provides a PostgreSQL database, and
is a service that Prisma has a lot of support for.

Later in the project when the application hosting was being considered, it was
decided that the author would the application on Vercel, which is a service that
provides hosting for NextJS applications. Vercel has a feature called `Vercel for
Databases', which allows you to create a PostgreSQL database that is hosted on
Vercel. It was a relatively simple process to migrate the database from Supabase
to Vercel, and the author was able to do this without too much difficulty. The
Vercel database is also a PostgreSQL database, which meant that the author could
continue to use Prisma with the database without any issues.

%%%%%%%%%%%% Implementation %%%%%%%%%%%%

\chapter{Implementation}\label{ch:implementation}

\section{Authentication}\label{sec:authentication}

\section{User Management}\label{sec:usermanagement}

\section{CRUD Operations}\label{sec:crudoperations}

\section{Data Loading}\label{sec:dataloading}

\section{Known Issues}\label{sec:knownissues}

%%%%%%%%%%%% Testing %%%%%%%%%%%%

\chapter{Testing}\label{ch:testing}

\section{Unit Testing}\label{sec:unittesting}

\section{User Testing}\label{sec:usertesting}

%%%%%%%%%%%% Project Planning %%%%%%%%%%%%

\chapter{Project Planning}\label{ch:projectplanning}

%%%%%%%%%%%% Conclusion %%%%%%%%%%%%

\chapter{Conclusions}\label{ch:conclusion}

\bibliographystyle{ieeetr}
\bibliography{report}

\end{document}
