\documentclass[12pt, a4paper,twoside]{report}

%% Every LaTeX document begins with a preamble, which loads packages and
%% defines various settings to make the document look right. Mostly,
%% you can ignore everything in this template before \begin{document} on
%% line 74

\usepackage{mathtools,amsthm} % Enable useful mathematical symbols/environments
\usepackage{graphicx} % Enable graphics
\usepackage{fancyhdr,titlesec,microtype} % enable various formatting commands
\usepackage[margin=2cm]{geometry} % Set margin size
\usepackage{palatino} % Set the font
\usepackage[latin1]{inputenc} % Allow you to input accents, umlauts and other characters
\usepackage[T1]{fontenc} % Lets LaTeX print a wider array of characters

\usepackage{xcolor} % Enable coloured elements
\definecolor{mypurple}{HTML}{622567} %%% Purple
\definecolor{myred}{HTML}{D55C19} %%%EssexOrange
\definecolor{myblue}{HTML}{007A87} %%%Seagrass

% For technical reasons, hyperref should be loaded after all other packages
\usepackage[colorlinks,linkcolor=myblue,citecolor=mypurple]{hyperref}

\renewcommand{\baselinestretch}{1.5} % 1.5 line spacing

% Define \begin{theorem}, \end{theorem}, etc.
\theoremstyle{plain} % The following will be italicised
\newtheorem{theorem}{Theorem}[chapter]

\theoremstyle{definition} % The following environments will not use italics
\newtheorem{definition}[theorem]{Definition}
\newtheorem{example}[theorem]{Example}

\numberwithin{equation}{chapter}

% Fancy headings
\pagestyle{fancy}
\setlength{\headheight}{15pt}
\fancyheadoffset[LE,RO]{0pt}
\renewcommand{\chaptermark}[1]{\markboth{#1}{}}
\renewcommand{\sectionmark}[1]{\markright{\thesection\ #1}}
\fancyhf{}
\fancyhead[LE]{\makebox[0pt][l]{\thepage}\hfill\leftmark}
\fancyhead[RO]{\rightmark\hfill\makebox[0pt][r]{\thepage}}
\fancypagestyle{plain}{%
    \fancyhead{} % get rid of headers
    \renewcommand{\headrulewidth}{0pt} % and the line
}

% Fancy chapter numbers
\titleformat{\chapter}[display]
    {\normalfont\bfseries\color{myred}}
    {\filleft\hspace*{-60pt}
        \rotatebox[origin=c]{90}{
            \normalfont\color{black}\large
            \textls[180]{\textsc{\chaptertitlename}}
        }
        \hspace{10pt}
        {\setlength\fboxsep{0pt}
            \colorbox{myred}{\parbox[c][2cm][c]{1.5cm}{
                    \centering\color{white}\fontsize{40}{45}\selectfont\thechapter}
            }
        }
    }
    {10pt}
    {\titlerule[2.5pt]\vskip3pt\titlerule\vskip4pt\LARGE\sffamily}

\begin{document} % Start your document

%%%%%%%%%%%% BEGIN TITLE PAGE %%%%%%%%%%%%

\thispagestyle{empty} % For the title page, no header / footer

\noindent
\begin{minipage}{0.1\textwidth}
    \includegraphics[height=4.5em]{essex.png}
\end{minipage}
\hfill
\begin{minipage}{0.5\textwidth}
    \begin{center}
        \renewcommand\familydefault{\sfdefault}
        \fontfamily{phv}\selectfont
        {\large School of Computer Science and Electrical Engineering}
    \end{center}
\end{minipage}

\begin{center}
    \noindent\textcolor{myred}{\rule{\linewidth}{4.8pt}}

    \vspace{2em}
    \noindent {\LARGE \sc Capstone Project Dissertation}

    \vspace{3em}
    \noindent {\Huge{\color{myblue} GoLearn - Learning Resource Management
            System}}

    \vspace{3em}
    \noindent {\Large \bf Bradley Beasley}

    \vfill
    \noindent {\Large {Supervisor:} {\color{mypurple} \bf Dr Alexandoros
            Voudoris}}

    \vspace{0.5em}
    \noindent\textcolor{myred}{\rule{\linewidth}{4.8pt}}

    \vspace{2em}
    {\Large \today }

    {\Large Colchester}
\end{center}

\clearpage

\abstract{
    There are plenty of services and programs out there for allowing Teachers
    to upload resources and for their learners to access them. Most of these
    are not purpose designed for places of learning, such as school, colleges
    and universities. Those that exist for this specific use-case tend to be
    over-engineered and can be overly complicated (Moodle is a good example
    of this). The purpose of GoLearn is to provide a simple LRM experience
    for teachers and students to interact with the same resources without a
    large amount of overhead in setting up the system: it should work out of
    the box.\\This report contains a look into methods of implementation for
    web applications in general, plus the choices in technology and design
    that were made for GoLearn.
}

%%%%%%%%%%%% END TITLE PAGE %%%%%%%%%%%%

\tableofcontents

%%%%%%%%%%%% Introduction %%%%%%%%%%%%

\chapter{Introduction}\label{ch:introduction}

%%%%%%%%%%%% Background %%%%%%%%%%%%

\chapter{Background}\label{ch:background}

\section{Similar Services}\label{sec:similarservices}

There are various LRMS's available in the market, with some of the more popular
one being:

\begin{description}
    \item[Udemy] {
        is a free to access learning site that puts a paywall over all of
        the courses. Creators can sign up to the site and create a course very
        quickly. Learners can sign up to courses (paying whatever charge is
        applied) and have unlimited access to them. They can access these from
        their own `Courses' dashboard.
    }
    \item[Skillshare] {
        is a similar service to Udemy, but is subscription based rather than
        pay per course. This means that learners can access all of the courses
        on the site for a monthly fee. Creators can create courses and upload
        them to the site, with the same access restrictions as Udemy.
    }
    \item[LearnDash] {
        is a WordPress plugin that allows you to create and sell
        courses on your WordPress site. It is a very popular plugin, with
        over 50,000 active installs and a 4.5-star rating on the WordPress
        plugin repository. It is a very powerful plugin, with a lot of
        features, but it is also very complex and can be difficult to use.
        \cite{learndash}\\
        Being a plugin for WP, it is very easy to customise how the plugin
        looks by modifying the theme on the site to best suit the business /
        individual's needs. The plugin allows teachers to create courses and
        students to access them, with a lot of customisation options for the
        courses.
    }
    \item[Moodle\cite{moodle}] {
        is the closest comparison that is currently available to GoLearn. It
        is a PHP based LMS that is very powerful and feature-rich. It allows 
        learning institutions to design their own courses/modules and add
        specific `student' users to them. This is a very similar model to
        what GoLearn is aiming to achieve. The main difference is that Moodle
        is very complex and can be difficult to use. It is also very
        resource-intensive, requiring a lot of server resources to run
        effectively. 
    }
\end{description}

\section{Tech Stacks}\label{sec:techstacks}

A `Tech Stack' is the term used to describe the combination of programming
languages, frameworks, libraries, and tools that are used to build a software
application. The choice of a tech stack is crucial as it can affect the
performance, scalability, and maintainability of the application. The following
are some of the popular tech stacks used in web development:

\begin{description}
    \item[LAMP/LEMP Stacks] {}
    \item[MEAN/MERN/MEVN Stacks] {}
\end{description}

\section{Databases and ORMs}\label{sec:databasesorms}

what is a database.
what is an ORM
popular databases and ORMs

%%%%%%%%%%%% Technical Specification %%%%%%%%%%%%

\chapter{Technical Specification}\label{ch:techspec}

\section{Framework}\label{sec:framework}

Which stack was chosen and why
which framework was chosen within that stack and why

\section{Database and ORM}\label{sec:databaseorm}

Which database was chosen and why
which ORM was chosen and why

%%%%%%%%%%%% Implementation %%%%%%%%%%%%

\chapter{Implementation}\label{ch:implementation}

\section{Authentication}\label{sec:authentication}

\section{User Management}\label{sec:usermanagement}

\section{CRUD Operations}\label{sec:crudoperations}

\section{Data Loading}\label{sec:dataloading}

\section{Known Issues}\label{sec:knownissues}

%%%%%%%%%%%% Testing %%%%%%%%%%%%

\chapter{Testing}\label{ch:testing}

\section{Unit Testing}\label{sec:unittesting}

\section{User Testing}\label{sec:usertesting}

%%%%%%%%%%%% Project Planning %%%%%%%%%%%%

\chapter {Project Planning}\label{ch:projectplanning}

%%%%%%%%%%%% Conclusion %%%%%%%%%%%%

\chapter{Conclusions}\label{ch:conclusion}

\bibliographystyle{ieeetr}
\bibliography{report}
\footnotetext{
    \textbf{Information}
    \emph{This document was written in LaTex and compiled with PDFTex and
        BibTex}
}

\end{document}
