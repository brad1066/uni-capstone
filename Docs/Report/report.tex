\documentclass[12pt, a4paper,twoside]{report}

%% Every LaTeX document begins with a preamble, which loads packages and
%% defines various settings to make the document look right. Mostly,
%% you can ignore everything in this template before \begin{document} on
%% line 74

\usepackage{mathtools,amsthm} % Enable useful mathematical symbols/environments
\usepackage{graphicx} % Enable graphics
\usepackage{fancyhdr,titlesec,microtype} % enable various formatting commands
\usepackage[margin=2cm]{geometry} % Set margin size
\usepackage{palatino} % Set the font
\usepackage[latin1]{inputenc} % Allow you to input accents, umlauts and other characters
\usepackage[T1]{fontenc} % Lets LaTeX print a wider array of characters

\usepackage{xcolor} % Enable coloured elements
\definecolor{mypurple}{HTML}{622567} %%% Purple
\definecolor{myred}{HTML}{D55C19} %%%EssexOrange
\definecolor{myblue}{HTML}{007A87} %%%Seagrass

% For technical reasons, hyperref should be loaded after all other packages
\usepackage[colorlinks,linkcolor=myblue,citecolor=mypurple]{hyperref}

\renewcommand{\baselinestretch}{1.5} % 1.5 line spacing

% Define \begin{theorem}, \end{theorem}, etc.
\theoremstyle{plain} % The following will be italicised
\newtheorem{theorem}{Theorem}[chapter]

\theoremstyle{definition} % The following environments will not use italics
\newtheorem{definition}[theorem]{Definition}
\newtheorem{example}[theorem]{Example}

\numberwithin{equation}{chapter}

% Fancy headings
\pagestyle{fancy}
\setlength{\headheight}{15pt}
\fancyheadoffset[LE,RO]{0pt}
\renewcommand{\chaptermark}[1]{\markboth{#1}{}}
\renewcommand{\sectionmark}[1]{\markright{\thesection\ #1}}
\fancyhf{}
\fancyhead[LE]{\makebox[0pt][l]{\thepage}\hfill\leftmark}
\fancyhead[RO]{\rightmark\hfill\makebox[0pt][r]{\thepage}}
\fancypagestyle{plain}{%
    \fancyhead{} % get rid of headers
    \renewcommand{\headrulewidth}{0pt} % and the line
}

% Fancy chapter numbers
\titleformat{\chapter}[display]
    {\normalfont\bfseries\color{myred}}
    {\filleft\hspace*{-60pt}
        \rotatebox[origin=c]{90}{
            \normalfont\color{black}\large
            \textls[180]{\textsc{\chaptertitlename}}
        }
        \hspace{10pt}
        {\setlength\fboxsep{0pt}
            \colorbox{myred}{\parbox[c][2cm][c]{1.5cm}{
                    \centering\color{white}\fontsize{40}{45}\selectfont\thechapter}
            }
        }
    }
    {10pt}
    {\titlerule[2.5pt]\vskip3pt\titlerule\vskip4pt\LARGE\sffamily}

\begin{document} % Start your document

%%%%%%%%%%%% BEGIN TITLE PAGE %%%%%%%%%%%%

\thispagestyle{empty} % For the title page, no header / footer

\noindent
\begin{minipage}{0.1\textwidth}
    \includegraphics[height=4.5em]{essex.png}
\end{minipage}
\hfill
\begin{minipage}{0.5\textwidth}
    \begin{center}
        \renewcommand\familydefault{\sfdefault}
        \fontfamily{phv}\selectfont
        {\large School of Computer Science and Electrical Engineering}
    \end{center}
\end{minipage}

\begin{center}
    \noindent\textcolor{myred}{\rule{\linewidth}{4.8pt}}

    \vspace{2em}
    \noindent {\LARGE \sc Capstone Project Dissertation}

    \vspace{3em}
    \noindent {\Huge{\color{myblue} GoLearn - Learning Resource Management
            System}}

    \vspace{3em}
    \noindent {\Large \bf Bradley Beasley}

    \vfill
    \noindent {\Large {Supervisor:} {\color{mypurple} \bf Dr Alexandoros
            Voudoris}}

    \vspace{0.5em}
    \noindent\textcolor{myred}{\rule{\linewidth}{4.8pt}}

    \vspace{2em}
    {\Large \today }

    {\Large Colchester}
\end{center}

\clearpage

\abstract{
    There are plenty of services and programs out there for allowing Teachers
    to upload resources and for their learners to access them. Most of these
    are not purpose designed for places of learning, such as school, colleges
    and universities. Those that exist for this specific use-case tend to be
    over-engineered and can be overly complicated (Moodle is a good example
    of this). The purpose of GoLearn is to provide a simple LRM experience
    for teachers and students to interact with the same resources without a
    large amount of overhead in setting up the system: it should work out of
    the box.\\This report contains a look into methods of implementation for
    web applications in general, plus the choices in technology and design
    that were made for GoLearn.
}

%%%%%%%%%%%% END TITLE PAGE %%%%%%%%%%%%

\tableofcontents

%%%%%%%%%%%% Introduction %%%%%%%%%%%%

\chapter{Introduction}\label{ch:introduction}

%%%%%%%%%%%% Background %%%%%%%%%%%%

\chapter{Background}\label{ch:background}

\section{Similar Services}\label{sec:similarservices}

There are various LRMS's available in the market, with some of the more popular
one being:

\begin{description}
    \item[Udemy] {
        Udemy is a free to access learning site that puts a paywall over all of
        the courses. Creators can sign up to the site and create a course very
        quickly. Learners can sign up to courses (paying whatever charge is
        applied) and have unlimited access to them. They can access these from
        their own `Courses' dashboard.\\
        Before putting together their first course, the creator must go through
        a short survey, which is likely to be used for internal statistic within
        the Udemy Business than for anything functional. It also serves top allow
        them to suggest some of their own resources to get the creator going based
        on their level of experiences in teaching, video creation and gaining an
        audience. Once this survey has been completed, the creator gains access
        to an instructor dashboard. Here they can see what courses they have,
        their status etc. It is also here that they start the creation of a new
        course (via an action button), and receive learning resource suggestions
        from Udemy.\\
        The process of making a new Course is very simple. The user is led through
        a multi-step form that focuses on different aspects of the general makeup
        of the course, and provides a good UX to do so. Once that form has been
        filled in, there is a full CMS provided for the creator to put together
        the working components of their course, such as Learning Outcomes,
        Requirements etc. It also provides a space for each component in the course
        to be added. Once the course has been fleshed out with content, it is then
        submitted for an internal review by Udemy.\\
        The way that the application UI and UX is designed is something that is
        going to inspire my own designs and application flow quite significantly.
        As the application that I am going to be making is not a SAAS product, and
        is going to be a working example of what can be provided to a teaching
        institution (such as the University of Essex), the separate parts of the
        app will only be accessible to those with related roles
        (Teacher/Student/Admin).\\
        My research for this service has been done solely through interacting with
        the site myself. This was possible as all aspects of the creation service
        are free, and I own a number of courses myself.
    }
    \item[Skillshare] {
        SkillShare offers a different business model to generate income from
        courses. Where Udemy uses a Course as a Product model, SkillShare
        provides a Subscription service to access all content that they provide.
        Teachers are then paid out based on their course metrics
        \cite{skillshare-teacher-earnings}. To summarise, they are paid out from
        a set teaching fund based on their engagement, total time spent on their
        content (with a minimum of 75 minutes across courses a month).\\
        The student experience of SkillShare is very similar to that of Udemy,
        providing a simple and clean UI/UX for them to interact with the service
        and classes. The Teacher experience is somewhat different in terms of
        signing up, but the actual course creation is very similar.
        \cite{skillshare-ux-1}\cite{skillshare-ux-2}
    }
    \item[LearnDash] {}
    \item[Moodle] {}
\end{description}

\section{Tech Stacks}\label{sec:techstacks}

A `Tech Stack' is the term used to describe the combination of programming
languages, frameworks, libraries, and tools that are used to build a software
application. The choice of a tech stack is crucial as it can affect the
performance, scalability, and maintainability of the application. The following
are some of the popular tech stacks used in web development:

\begin{description}
    \item[LAMP/LEMP Stacks] {}
    \item[MEAN/MERN/MEVN Stacks] {}
\end{description}

\section{Databases and ORMs}\label{sec:databasesorms}

what is a database.
what is an ORM
popular databases and ORMs

%%%%%%%%%%%% Technical Specification %%%%%%%%%%%%

\chapter{Technical Specification}\label{ch:techspec}

\section{Framework}\label{sec:framework}

Which stack was chosen and why
which framework was chosen within that stack and why

\section{Database and ORM}\label{sec:databaseorm}

Which database was chosen and why
which ORM was chosen and why

%%%%%%%%%%%% Implementation %%%%%%%%%%%%

\chapter{Implementation}\label{ch:implementation}

\section{Authentication}\label{sec:authentication}

\section{User Management}\label{sec:usermanagement}

\section{CRUD Operations}\label{sec:crudoperations}

\section{Data Loading}\label{sec:dataloading}

\section{Known Issues}\label{sec:knownissues}

%%%%%%%%%%%% Testing %%%%%%%%%%%%

\chapter{Testing}\label{ch:testing}

\section{Unit Testing}\label{sec:unittesting}

\section{User Testing}\label{sec:usertesting}

%%%%%%%%%%%% Project Planning %%%%%%%%%%%%

\chapter {Project Planning}\label{ch:projectplanning}

%%%%%%%%%%%% Conclusion %%%%%%%%%%%%

\chapter{Conclusions}\label{ch:conclusion}

\bibliographystyle{ieeetr}
\bibliography{report}
\footnotetext{
    \textbf{Information}
    \emph{This document was written in LaTex and compiled with PDFTex and
        BibTex}
}

\end{document}
